\chapter{Principali patologie geriatriche}
\section{Valutazione e presa in carico globale}
L'esperienza di oggi si basa su un caso clinico di un paziente con sclerosi 
multipla.
La sclerosi multipla è una malattia cronica demielinizzante a carico del SNC.
Colpisce di solito il giovane adulto, dai 20 ai 40 anni. A seconda delle aree 
colpite, i deficit possono essere molto diversi, a carico del sistema 
muscolo scheletrico, visivi, cognitivi, di disfagia\dots

L'approccio globale, però, può essere applicato in maniera trasversale a 
tutti 
i pazienti, in particolare a quelli con malattie croniche. La fragilità 
dell'anziano viene determinata dalla complessità del caso clinico, ma anche 
dallo svantaggio rispetto all'ambiente e al ruolo sociale determinato dalla 
malattia. Avendo a che fare con questi pazienti, l'intervento è sul soggetto, 
ma anche coi vari famigliari e caregiver, e diventa particolarmente importante 
l'approccio multiprofessionale, e dovrebbe esserci una stretta collaborazione, 
perché con questi pazienti è impossibile capire bene il vissuto di malattia 
del paziente senza questo approccio. Il lavoro in équipe permette una visione 
comune dei problemi, e una maggiore motivazione del paziente.

Con questi pazienti non è possibile concentrarsi solo sulla dimensione motoria 
e funzionale, ma bisogna mettersi a pensare a tutti gli aspetti che 
caratterizzano il soggetto come persona, per dargli la possibilità di convivere 
con la sua dimensione e di accettarla. Si deve, insomma, lavorare sulla sua 
qualità di vita.

Il paziente, nello specifico, ha 33 anni, convive da 5 anni, ha un fratello 
maggiore, un padre col Parkinson, e la madre à morta. Veniva seguito in modo 
ambulatoriale, due volte a settimana. Il rapporto con la compagna sembrava 
teso. Ha subito interventi urologici e all'apparato riproduttivo. Negli anni 
scorsi ha avuto vari deficit, dapprima parestesie, poi deficit visivi e dolori 
neuropatici. In seguito sono comparsi i deficit sfinterici e giramenti di 
testa. I deficit sfinterici sono molto importanti, sia con perdite anche 
complete, sia con difficoltà alla minzione al momento effettivo dello 
svuotamento. I deficit includevano, parzialmente, anche la parte fecale. Era 
costretto ad usare un panno sia di giorno sia di notte, quindi dopo un ciclo di 
riabilitazione perineale è stato addestrato al cateterismo. Questo ha permesso 
di smettere il panno e di guadagnare una discreta autonomia, ma ha portato ad 
alcune infezioni. Ha disturbi sessuali, di ansia, facile faticabilità e deficit 
di equilibrio.

Altri aspetti sono tremori intenzionali, soprattutto a destra, ma anche agli 
arti inferiori. pur essendo destro, il paziente si trovava quindi a fare molte 
cose con la sinistra. L'urgenza minzionale lo portava a interrompere la terapia 
anche due volte in 40 minuti. tutte le posture vengono raggiunte e mantenute 
senza difficoltà, i passaggi posturali sono in autonomia e senza difficoltà, 
la mobilità articolare e la forza muscolare sono buone, tranne che per la 
faticabilità. La deambulazione è in autonomia per circa un chilometro, poi 
compaiono i dolori e la faticabilità.

Dal punto di vista cognitivo non c'erano disturbi, e anche la comunicazione era 
buona. Da un punto di vista emotivo, il paziente era consapevole della 
malattia, ma al momento di raccontare la storia clinica ha omesso completamente 
la parte che riguardava l'apparato genito-urinario. In palestra ha sempre 
mantenuto un atteggiamento positivo e collaborante. Da un colloquio con la 
psicologa era emerso un disturbo di ansia, che ha portato a qualche seduta, che 
poi sono state interrotte. In queste sedute è emerso che prima dei sintomi che 
hanno portato agli accertamenti ne aveva avuti altri, che però ha ignorato. 
L'ansia lo porta a ridurre sempre più le attività sociali, con contrasto con 
la compagna, che vorrebbe avere una vita normale. Verso il fratello maggiore prova 
invidia, e non nomina mai la famiglia spontaneamente. Parla con disagio anche 
del suo lavoro, dove è passato da operativo a logistico. La psicologa lo 
definisce non sempre collaborante, ma in palestra l'atteggiamento è sempre 
stato buono.

Gli elementi positivi sono la giovane età (motivazione, risorse fisiche), la 
collaborazione, l'autonomia delle ADL e le funzioni cognitive integre. I 
negativi sono la presenza di dolore e tremore, la rete sociale poco sviluppata, 
l'ansia e il cateterismo.

I principali problemi riabilitativi sono:
\begin{itemize}
\item il deficit di equilibrio
\item la resistenza e la fatica
\item la continenza e il controllo sfinterico
\end{itemize}
dopo questa analisi, si capisce come il paziente anziano abbia in realtà molti 
tratti in comune con il paziente con malattie croniche e degenerative, come la 
numerosità dei sistemi funzionali coinvolti, la variabilità, la fragilità e 
complessità, i bisogni e le aspettative diversi dal classico paziente della 
stessa età, e la ridotta spinta motivazionale: esistono pazienti che vedono 
anche i piccoli miglioramenti in modo positivo e stimolante, e altri che, 
sapendo che il miglioramento positivo è piccolo, si demoralizzano.

La valutazione multidimensionale permette di darci una visione più completa di 
quella che avremmo vedendo il paziente solo in palestra, e permette anche di 
migliorare la relazione col paziente: avendo tempi di trattamento molto lunghi, 
creare il rapporto, con fiducia e motivazione, è utile anche per noi, oltre 
che per il paziente. La valutazione multidimensionale permette di lavorare 
meglio anche sul potenziale residuo, di mettere meglio in relazione i vari 
dati, 
e di stabilirne una gerarchia. Il danno deve essere quindi ridotto, la 
sofferenza alleviata, le risorse devono essere sostenute e la qualità di vita 
va migliorata, per permettere al paziente di vivere nella maniera migliore 
possibile la vita rispetto alle sue possibilità.

\section{Il morbo di Parkinson}
\subsection{Slide}
Il morbo di Parkinson è stato descritto nel 1817, ed è una malattia
neurodegenerativa molto frequente. Non se ne conosce l'eziologia, ma in generale
fa parte di un gruppo più ampio, le \textit{sindromi parkinsoniane}. La sua
prevalenza aumenta con l'età. La malattia colpisce i neuroni dopaminergici della
\textit{substantia nigra}, e il danno compare quando almeno il 60 percento dei
neuroni è danneggiato. In seguito vengono colpiti anche altri sistemi, come
quello serotoninergico, noradrenalinergico e acetilcolinergico, il che porta
alla comparsa di sintomi non motori. A causa della sua variabilità e della
complessità, è necessario un approccio multidisciplinare e multidimensionale,
tagliato sul paziente.

I sintomi si dividono in
\begin{itemize}
\item Motori, come:
	\begin{itemize}
	\item Bradicinesia: è una riduzione della velocità e dell'ampiezza del
	movimento. \'E fondamentale per arrivare alla diagnosi, e compromette molto le
	abilità del paziente nelle ADL. Può essere testata con il
	\textit{finger tapping}, e porta anche alla riduzione dell'ampiezza e della
	lunghezza del passo, ad una riduzione della mimica facciale, e ad una fissità
	dello sguardo. Si interviene con i \textit{cues}, che possono essere sia visivi
	che uditivi.
	\item Tremore a riposo a bassa frequenza
	\item Ipertono che porta a rigidità: è una resistenza allo stiramento passivo,
	con genesi multifattoriale. \'E del tipo "tubo di piombo", anziché quella a
	"troclea dentata" tipica degli emiplegici (colpisce sia gli agonisti che gli
	antagonisti), e porta, nel tempo, a rigidità e accorciamenti muscolo-tendinei.
	Si interviene migliorando le capacità di rilassamento, portando a livello di
	coscienza il problema, ed effettuando stretching per migliorare l'elasticità
	dei tessuti muscolotendinei.
	\item Instabilità posturale: anche in questo caso, la genesi è multifattoriale,
	e il problema può essere testato con uno specifico \textit{pull test}, in cui
	si tira il paziente verso di sé, standogli alle spalle, e si osservano le
	reazioni. Questo problema porta ad instabilità e cadute del paziente.
	Il nostro intervento si articola con esercizi di equilibrio sia in statica che
	nel cammino, e con la pratica del \textit{tai-chi}.
	\end{itemize}
\item Non motori, come disturbi dell'umore, del comportamento, cognitivi e
sensoriali (con aumento della percezione del dolore), e disfunzioni del SN
autonomo, con sbalzi della pressione arteriosa, e aumentata salivazione e
secrezione di sebo.
\item Complicanze secondarie, di solito legate all'assunzione di L-DOPA, infatti
aumentano con l'aumentare del tempo di assunzione.
	\begin{itemize}
	\item Fluttuazioni motorie:
		\begin{itemize}
		\item Perdita di risposta: la risposta alla somministrazione di L-DOPA può
		essere ritardata, o può non comparire una risposta con una singola dose, o può
		calare la risposta, verso la fine dell'effetto della dose.
		\item Fenomeno ON-OFF: assieme all'OFF motorio, possono comparire problemi
		sensitivi, psichici, cognitivi e autonomici.
		\item Freezing: presente sia in fase ON, che, più frequentemente, in fase OFF.
		\end{itemize}
	\item Movimenti involontari: sono movimenti che compaiono con l'assunzione
	dell'L-DOPA, e possono disturbare il soggetto, a volte anche portare dolore,
	con crampi e distonia. I movimenti sono di tipo coreiforme, e possono
	interessare il capo, gli arti, il busto, e, a volte, i muscoli della
	respirazione. Possono essere
		\begin{itemize}
		\item Discinesie disfasiche, che compaiono all'inizio e alla fine della
		risposta all'L-DOPA.
		\item Discinesie di picco, che compaiono dopo circa un'ora dall'assunzione.
		\end{itemize}
	A volte, questi fenomeni si possono sommare temporalmente, e possono arrivare
	ad essere presenti per tutta la durata del ciclo. Esistono anche le posture
	distoniche, come sindrome di Pisa e antecollo, in fase avanzata, che si
	controllano bene con la fisioterapia.
	\end{itemize}
\end{itemize}

La rigidità, la bradicinesia e l'instabilità, assieme all'età solitamente
avanzata, portano a postura flessa, disequilibrio, retrazioni articolari,
ipostenia, riduzione dell'attività, e decondizionamento. \'E importante anche
ricordare che è possibile una certa comorbilità, grazie all'età non proprio
giovanissima.

L'intervento del fisioterapista si basa su un intervento precoce, fin dalle
fasi 1 e 2, che si deve diversificare in base alla stadiazione della malattia e
alle esigenze del singolo malato. L'esercizio fisico è fondamentale e deve
essere costante, e il paziente e il caregiver vanno addestrati a gestire i
deficit motori.

\subsubsection{La valutazione}

Gli stadi della malattia sono, secondo la scala di Hoehn e Yahr:
\begin{description}
\item [Stadio 1] Malattia unilaterale
\item [Stadio 1.5] Malattia unilaterale con coinvolgimento assiale
\item [Stadio 2] Malattia bilaterale, senza problemi di equilibrio
\item [Stadio 2.5] Malattia bilaterale, l'equilibrio viene recuperato dopo il
\textit{pull test}
\item [Stadio 3] Disabilità lieve o moderata, con qualche problema di equilibrio
\item [Stadio 4] Disabilità marcata, ma cammino conservato
\item [Stadio 5] Paziente in sedia a rotelle o a letto
\end{description}

La scala di solito più usata è la UPDRS, ossia \textit{unified parkinson's
disease rating scale}, che tiene conto di:
\begin{itemize}
\item Aspetti cognitivi e affettivi
\item ADL, dividendole sia in fase ON che OFF. Vengono valutati aspetti di
linguaggio, salivazione, deglutizione, scrittura, manualità con gli utensili,
capacità di vestirsi, mantenere l'igiene personale, girarsi nel letto e
aggiustare le coperte, aspetti delle cadute, del freezing, della marcia,
del tremore, e i vari disturbi sensoriali legati al Parkinson.
\item Esame motorio, tenendo conto del linguaggio, dell'espressione del volto,
del tremore a riposo, dell'attività delle mani, della rigidità, del
picchiettamento delle dita, del movimento delle mani, dell'agilità delle gambe,
la postura, l'andatura, la stabilità posturale, la bradicinesia e l'ipocinesia
corporea.
\item Complicanze della terapia
\end{itemize}

\textbf{Gli obiettivi negli stadi 1 e 2} sono di prevenire il ritiro sociale e 
l'inattività, mantenere una buona forma aerobica, flessibilità e forza muscolare,
e di promuovere l'educazione terapeutica. \textbf{Gli interventi} che possono
essere applicati sono: di promuovere l'attività fisica regolare, in particolar
modo se permette di socializzare, come nel caso del nuoto, della danza e del tai
chi, e di promuovere incontri informativi e formativi sulla malattia.

\textbf{Negli stadi che vanno dal 2.5 al 4}, gli obiettivi sono come i
precedenti, a cui si aggiungono la riduzione dell'instabilità e la prevenzione
delle cadute, l'apprendimento di strategie per gestire la fase OFF, e
l'addestramento del caregiver, estendendo la fisioterapia all'ambiente
domestico. \textbf{Gli interventi} sono di promuovere l'attività fisica
regolare, rieducare all'equilibrio e tenere un diario delle cadute, ed iniziare
a introdurre dei \textit{cues} uditivi e visivi, delle strategie ambientali, e 
delle modifiche ambientali a domicilio.

\textbf{Gli obiettivi nello stadio 5} sono quelli precedenti, a cui si
aggiungono il mantenimento delle funzioni vitali, la prevenzione o riduzione
dei danni secondari, come retrazioni, deformità e lezioni da pressione, e
l'addestramento del caregiver. \textit{Gli interventi} sono una mobilizzazione
assistita globale, il mantenimento delle ADL residue, e l'educazione del
caregiver nel posizionamento e nella movimentazione del paziente e degli ausili.

\paragraph{Take home messages}
La malattia di Parkinson è una patologia cronica neurodegenerativa con frequenza
elevata. \'E una malattia complessa, multiforme, i pazienti sono molto
variabili. Necessita di un approccio multiprofessionale e multidimensionale,
che deve essere diversificato per ogni paziente e per ogni stadio della malattia.
I sintomi caratteristici sono il tremore a bassa frequenza, la bradicinesia, la
rigidità a tubo di piombo, che portano a instabilità posturale.
Vengono coinvolti anche altri sistemi, il che porta a sintomi non motori,
secondari, che aumentano la disabilità.

\subsection{Appunti}
\'E una patologia cronica neurodegenerativa. Colpisce i neuroni dopaminergici 
della \textit{substantia nigra}. Vengono colpiti anche i sistemi 
muscoloscheletrico, Sistema nervoso autonomo, con aumentata produzione sebacea 
e di saliva, sistema escretorio, con urgenza minzionale e ritenzione, 
depressione, decadimento cognitivo, problemi di motilità intestinale. I 
pazienti tendono a perdere la vita sociale, perché si vergognano della
condizione.

I segni caratteristici sono: tremore a bassa frequenza, rigidità a dente di 
sega, bradicinesia. Questi portano, combinati, all'instabilità posturale nelle 
fasi meno avanzate.

La malattia diventa sintomatica solo dopo che il 60\% dei neuroni 
dopaminergici viene colpito dalla degenerazione. questa è l'ennesima prova 
della ridondanza, del numero di neuroni molto più elevato del necessario. 
Questo fenomeno è uno di quelli su cui si basa la riabilitazione neurologica, 
sul fatto che, per quanto ci sia un danno al nostro cervello, è possibile 
attivare delle nuove risorse che fino a quel momento erano rimaste inutilizzate.

Il danno dopaminergico viene trattato con la L-DOPA, e la depressione è 
dovuta, oltre che all'abbassamento dell'umore ovvio per la presenza della 
malattia, anche al danno al sistema serotoninergico.
Tra i sintomi non motori, aumenta la percezione del dolore, per l'abbassamento 
dell'umore. Emerge anche una difficoltà di percezione a livello profondo, 
propriocettivo, che porta anche ad un abbassamento dell'equilibrio.

Nelle complicanze secondarie, evidenziamo come le complicanze motorie seguano 
il ciclo on-off dei farmaci. La L-DOPA, col tempo, va incontro a ritardato 
assorbimento o a esaurimento anticipato. Si presentano anche dei sintomi di 
tipo liberatorio, discinesie di tipo coreiforme, e distonie (tipiche 
all'alluce, 
che si pone come col Babinski, al collo, che porta il paziente in 
anteposizione, 
e al tronco, con postura inclinata) che si presentano in fase on. Il tremore 
non viene controllato con la L-DOPA, ma si può controllare con degli 
interventi di neurochirurgia. Le distonie vengono trattate con stretching, mentre i 
movimenti coreiformi con rilassamento, cercando di introdurre un aspetto 
cognitivo nel controllo. Per la distonia del collo si utilizza la tossina 
botulinica, ma si va incontro alla possibilità di aumentare la disfagia. 

Altri danni possono essere le fratture da cadute, le retrazioni e l'anchilosi 
per la rigidità, l'atrofia e l'allettamento, con tutte le conseguenze tipiche, 
per l'ipocinesia.

La bradicinesia è responsabile della riduzione dell'ampiezza e della lunghezza 
del passo, dell'ipomimia facciale, e della fissità dello sguardo. Il 
trattamento viene eseguito tramite dei \textit{cues}, visivi o uditivi, che 
vanno a sostituire gli automatismi del movimento, che vengono persi. Questi 
aiuti, soprattutto uditivi, necessitano della completa attenzione del paziente, 
che rende piuttosto difficile poter fare qualsiasi altra cosa nel frattempo. 
Uno dei problemi più grandi per i parkinsoniani è il \textit{dual tasking}, 
testabile con il test \textit{stop walking when talking}. Questo deficit 
discende da problemi nell'attenzione divisa. L'ipomimia porta alla facies 
fissa, con sorriso appena accennato, ridottissima mimica oculare. Hanno anche 
problemi a riconoscere le emozioni altrui, a causa di problemi nei neuroni 
specchio, e addirittura l'intensità nella percezione delle emozioni è più 
bassa.

La rigidità discende da un ipertono, quindi da una alterata risposta tonica dei 
muscoli. L'ipertono è di tipo co-contrattivo, perché risultano contratti sia 
gli agonisti che gli antagonisti, soprattutto nella muscolatura assiale, 
portando a un atteggiamento detto \textit{campto cormico}. L'ipertono non è, 
come quello piramidale, a serramanico, ma a dente di sega, o a troclea dentata.
Il tipo particolare di rilasciamento dell'ipertono potrebbe essere dovuto alla
sovrapposizione del tremore con l'ipertono co-contrattivo, anche se non se ne ha
certezza.
La rigidità porta, nel tempo, a rigidità articolari, accorciamenti
muscolo-tendinei, e deformità. Nel tempo, per l'atteggiamento cifotico del
dorso, si struttura una iperestensione del collo, per mantenere orizzontale lo
sguardo. Gli strumenti del fisioterapista, per la rigidità, sono: stretching,
per evitare le retrazioni, e presa di coscienza e miglioramento del
rilassamento, che però non sempre è facile o utile per il paziente.

L'instabilità posturale viene testata con il \textit{pull test}, nel quale si 
tira il paziente verso di sé, dalle spalle. Possono mettere in atto varie 
strategie per restare in piedi, come una strategia a livello della caviglia, 
dell'anca, o di un passo indietro. Se il paziente ha molta paura, la spinta si 
esegue da davanti, dallo sterno, con le nocche. Non ha una genesi 
monofattoriale, è responsabile, spesso, della riduzione della mobilità, delle 
cadute, e quindi, a volte, di fratture.
Le strategie di intervento del fisioterapista, in questo ambito, possono essere 
eseguite in statica o in dinamica, anche durante il cammino. Si può 
utilizzare il Thai-Chi, che ha delle evidenze scientifiche di efficacia.

L'intervento deve essere precoce, soprattutto in fase iniziale si procede con 
esercizio fisico, non per forza fisioterapia. \'E importante, in fase avanzata 
e dopo anni dall'esordio della malattia, procedere sempre più all'educazione 
terapeutica del caregiver, per insegnare strategie di mobilizzazione e di 
superamento del deficit di mobilità. 

Gli interventi nelle fasi avanzate sono gli stessi delle fasi precedenti, a cui 
si sommano la riduzione dell'instabilità posturale e la prevenzione delle 
cadute, con un maggiore intervento educativo al caregiver, per posizionare gli 
ausili e gestire in modo efficace le fasi OFF. Il terapista inizia ad agire 
sempre di più in casa, introducendo anche modificazioni all'ambiente domestico.

Negli stadi finali, si passa addirittura al mantenimento delle funzioni vitali, 
perché a causa delle modificazioni posturali, ad esempio, diventa difficile 
mantenere la respirazione, e si cerca di prevenire i danni secondari, come 
retrazioni, deformità e lesioni da pressione. Per farlo, si usano una 
mobilizzazione globale e assistita, si cerca di mantenere il più possibile 
compiti funzionali come le ADL, e si cerca di addestrare il caregiver a 
posizionare e movimentare i vari ausili. \'E importante ricordare che ogni 
paziente è particolare, a sé, quindi il trattamento deve essere sempre molto 
individualizzato, in modo da ottenere il massimo possibile da ogni paziente, 
sfruttando le potenzialità residue.

\section{Perturbazioni dell'equilibrio}
I gesti motori sono composti da una componente posturale, e da una volontaria.
La componente posturale permette di mantenere l'equilibrio, mentre la volontaria
di compiere il movimento vero e proprio. Il movimento, quindi, è una continua
interazione di individuo, compito e ambiente.

L'equilibrio è la capacità di mantenere l'allineamento tra il centro di massa
del corpo e la base d'appoggio, in modo da garantire la stabilità. Il controllo
posturale è l'attività deputata a mantenere questo allineamento, ed è quindi
parte integrante del movimento.
La stabilità può essere messa a rischio da forze esterne, che possono essere
dirette a noi, come una spinta, o essere della base di supporto, come il
movimento del mezzo su cui ci appoggiamo, o interne, come i movimenti volontari.

Gli scopi del controllo posturale sono:
\begin{itemize}
\item Sostenere il corpo contro gravità
\item Mantenere l'allineamento tra il centro di massa e la base d'appoggio
\item Stabilizzare alcune parti del corpo mentre altre si muovono
\end{itemize}

Il controllo posturale viene mantenuto grazie a \textbf{componenti neurali},
come le informazioni sensoriali che raccogliamo, la programmazione del
movimento, e il comando motorio efferente, e a \textbf{componenti
muscoloscheletriche}, come le caratteristiche delle articolazioni, dei muscoli
e dei tendini, che fungono da vincoli.

L'equilibrio può andare incontro a due tipi di perturbazioni: quelle previste,
che sono le più abituali, generano un adattamento anticipatorio (attivato 50
millisecondi prima della contrazione per compiere l'azione), mentre quelle
impreviste, che si verificano solo in situazioni insolite, sono di tipo
reattivo. Tutte queste reazioni possono essere ottimizzate attraverso
l'esercizio.

Le strategie per mantenere l'equilibrio sono:
\begin{itemize}
\item Motorie: organizzate in senso disto-prossimale
	\begin{description}
	\item[Strategie di caviglia] La flessione dorsale o plantare della caviglia.
	Funzionano in seguito a spinte con scarsa ampiezza o intensità.
	\item[Strategie di anca] La flessione e l'estensione dell'anca entrano in azione
	in seguito a perturbazioni più ampie o rapide, con una base d'appoggio più
	ristretta o cedevole.
	\item[Stepping action] Entra in azione quando le altre strategie falliscono.
	\end{description}
\item Sensoriali:
	\begin{description}
	\item[Controllo visivo] Tramite dei riferimenti esterni permette di mantenere
	un senso di verticalità.
	\item[Controllo tattile] Le afferenze dei piedi sono molto importanti, come la
	funzione \textit{aptica} della mano, quando regge il bastone o sfiora i mobili,
	per mantenere l'equilibrio.
	\item[Vestibolare] \'E fondamentale per il controllo posturale ad alta
	frequenza e ad alta velocità di movimento, permette di ridurre le oscillazioni
	del tronco, e trasmette informazioni sulla posizione del capo e del tronco
	nello spazio, permettendo allo stesso tempo la separazione tra la coordinazione
	del tronco e degli arti inferiori, quella del tronco e del capo, e quella del
	centro di massa corporea.
	\end{description}
\end{itemize}

Le alterazioni del controllo posturale, quindi, possono essere dovute ad
\begin{itemize}
\item Alterazioni del sistema muscolo scheletrico:
	\begin{itemize}
	\item Alterata distribuzione del carico, che può essere dovuta a patologie
	diverse, ad esempio cerebellari, ictus, artrosi, alterazione del ROM. Questi
	fanno sì che ci sia bisogno di strategie diverse dalla norma per mantenere
	o recuperare l'equilibrio, strategie che saranno più dispendiose rispetto a
	quelle normali, e che permetteranno una variabilità minore. Per questo, a volte
	le ortesi possono risultare controproducenti: lo schema già alterato non si
	adatta a dovere.
	\item Deficit di forza, che può influenzare le cadute.
	\end{itemize}
\item Alterazioni del controllo neuromotorio
	\begin{enumerate}
	\item Può cambiare l'ordine nel reclutamento dei muscoli, per le risposte
	posturali (anziché in senso disto-prossimale, avvenire a partire, magari, dal
	collo).
	\item Può esserci un ritardo nell'attivazione
	\item Può esserci una risposta eccessiva, con scarsa capacità di modulare la
	forza.
	\item Può esserci un adattamento ridotto, con presenza di risposte
	stereotipate.
	\item Si può perdere il controllo anticipatorio.
	\end{enumerate}
\item Disordini sensitivi: a carico di uno o più canali, contribuiscono a creare
una differenza tra i limiti di stabilità acquisiti in seguito ad una lesione, e
la loro rappresentazione interna. Questa incongruenza aumenta il rischio di
cadute.
\end{itemize}

L'invecchiamento influenza l'equilibrio in modi complessi e poco chiari: ad
esempio, nella popolazione anziana, si ha un aumento del rischio di cadute, con
un'origine multifattoriale. Questo rischio è maggiore nelle donne.
I fattori che contribuiscono ad aumentare questo rischio sono sia estrinseci,
quindi legati all'ambiente, sia intrinseci, quindi legati al processo
fisiologico di invecchiamento, sia legati alle varie disabilità che possono
essere acquisite nel tempo.
In particolare, si assiste ad una modifica
\begin{itemize}
\item Delle componenti muscoloscheletriche, con riduzione della forza, e aumento
dei vincoli articolari.
\item Delle componenti neurali, con
	\begin{itemize}
	\item Alterazione del controllo visivo, con contemporaneo aumento
	dell'importanza di questo canale, dovuto alla diminuzione dell'efficienza del
	controllo tattile e vestibolare.
	\item Diminuzione del controllo tattile, con riduzione dei recettori cutanei, e
	l'instaurarsi di disturbi sensitivi legati a patologie neurologiche centrali e
	periferiche.
	\item Riduzione del controllo vestibolare, per una degenerazione degli otoliti,
	delle ciglia e dei neuroni vestibolari. Questo porta all'aumento
	dell'incertezza nell'interpretare informazioni provenienti dai vari sistemi, e
	potrebbe spiegare il senso di vertigine e instabilità delle gambe che
	riferiscono alcuni anziani.
	\item Ritardo nell'attivazione della risposta efferente, che porta a una
	difficoltà nell'adottare provvedimenti anticipatori.
	\end{itemize}
\end{itemize}
Queste modifiche portano, associate, ad una modifica delle strategie di
adattamento, ad esempio si usa più la strategia d'anca che quella di caviglia,
e aumentano le oscillazioni posturali.

In risposta alle modifiche, si possono adattare dei \textbf{compensi}, che vanno
valorizzati, se portano ad un recupero di autonomia. I più diffusi sono:
\begin{itemize}
\item Allargare la base d'appoggio, sia da seduti che in stazione eretta,
abducendo gli arti inferiori ed extraruotandoli.
\item Usare le mani per appoggiarsi: permette di allargare la base d'appoggio,
modificando l'attivazione muscolare e i pattern di movimento degli arti
inferiori e del tronco. Se c'è una patologia respiratoria in atto può essere
controindicato l'atteggiamento in chiusura della gabbia toracica.
\item Lo spostamento del carico sull'arto meno colpito permette di modificare
gli aspetti di spazio-tempo del cammino, portando ad una più corretta
distribuzione del carico.
\item L'irrigidimento del tronco viene messo in atto con una co-contrazione di
agonisti e antagonisti, per ridurre le oscillazioni, spesso con poca efficacia,
e con una riduzione della flessibilità nelle risposte posturali.
\item Evitare le situazioni di rischio, reali e presunte, può essere pericoloso:
di solito viene adottato da chi è già caduto, e ha paura che succeda di nuovo.
I pazienti evitano molte delle ADL, e possono andare incontro ad una sindrome da
immobilizzazione.
\end{itemize}

Per valutare l'equilibrio esistono vari strumenti validati, la cui scelta
dipende dal setting, dal tempo e dal materiale che abbiamo a disposizione, dalle
nostre conoscenze, e dal tipo di paziente. La valutazione permette anche di fare
emergere il livello del rischio di caduta, ma questi strumenti non vanno confusi
con quelli per la valutazione del rischio di caduta usati dagli infermieri.

\subsection{Il trattamento}
Gli obiettivi del trattamento sono innanzitutto ridurre, recuperare o prevenire
le limitazioni che possono interferire con l'equilibrio, come la perdita di
forza e l'atteggiamento in equinismo della caviglia, per poi sviluppare
strategie, sensoriali e motorie, per il controllo posturale. Un altro obiettivo
è ampliare la gamma dei compiti e delle situazioni in cui si sappia controllare
l'equilibrio, soprattutto esercitandosi in situazioni coerenti con la vita di
tutti i giorni.

L'intervento è sistematico: si interviene a livello muscolo-scheletrico, con un
aumento della forza e della stabilità, a livello delle strategie, cercando
compensi o potenziamenti delle strategie già in atto, e a livello dei compiti
e dell'ambiente circostante, sia per allenare il soggetto, sia per fornire delle
facilitazioni. L'intervento sulle strategie si mette in atto con esercizi per
escludere la vista e aumentare le afferenze vestibolari e propriocettive, per
potenziare le risposte paracadute e la strategia d'anca, e promuovendo la
\textit{stepping action}. L'intervento sui compiti e sull'ambiente è, ad
esempio, la progressione dalla stazione seduta a quella eretta, usando sostegni,
che possono essere fissi o mobili, allargando o riducendo la base d'appoggio,
anche in posizioni strane dei piedi, ed eventualmente inserendo degli elementi
di difficoltà percettiva.

Nell'anziano, i disturbi del cammino sono tipici dell'invecchiamento, e
aumentano con l'andare avanti dell'età. Hanno un forte significato prognostico
per il declino funzionale e la morte.

\subsection{Il cammino}
Il cammino, in generale, è un pattern complesso, in cui interagiscono più
sistemi. Permette di spostarci in sicurezza, mantenendo l'equilibrio e
permettendoci di interagire con oggetti distanti. Ha un certo costo energetico,
che viene aumentato da tutte le alterazioni a cui può andare incontro.

\'E controllato da componenti neurali, come il generatore spinale del pattern
locomotorio, il controllo della corteccia e di circuiti sottocorticali, e dalle
informazioni sensoriali, da componenti muscolo-scheletriche, come le
articolazioni, i muscoli e i tendini, e dall'apparato cardiaco e circolatorio.

I disturbi che lo influenzano possono essere di:
\begin{itemize}
\item Basso livello: se il danno è periferico, dei sistemi sensoriale, muscolare
o osteoarticolare, o del SNP.
\item Medio livello: se il danno è nel SNC, generando un disturbo caratteristico
a seconda della sede della lesione.
\item Alto livello: se il danno nel SNC è diffuso, compromettendo vari circuiti,
corticali e sottocorticali, creando un disturbo generalizzato.
\end{itemize}

Nell'anziano, il cammino si modifica sia per la velocità, sia per la fase di
doppio appoggio ridotta, ma anche per i parametri spaziali, con una riduzione
della lunghezza del passo, e variazioni anche nelle dita del piede.

I principali valori per analizzare il cammino sono la velocità media, la
lunghezza media, la larghezza della base d'appoggio, e l'altezza media del
passo. Più questi valori variano nel tempo, nello stesso soggetto, peggiore sarà
la prognosi. Al contrario, un discostarsi del soggetto dai valori medi non ha un
grande significato prognostico.

Le modifiche del cammino sono correlate ad alcune modifiche legate
all'invecchiamento: diminuisce la forza dei muscoli, che però porta ad un calo
della velocità solo se è molto evidente, e diminuisce il ROM delle
articolazioni, che però porta ad una variazione del passo solo se è molto
marcata.

L'andatura dell'anziano, senza patologie associate, si caratterizza per una base
d'appoggio allargata, con passi corti, poca oscillazione delle braccia, postura
flessa, ridotta flessione delle anche e delle ginocchia, incertezza nei cambi di
direzione e facilità a cadere. Questa andatura ci fa capire che ci sono
disturbi, anche se non clinici, del SNC.

Gli obiettivi del nostro trattamento, in questo caso, sono di recuperare,
ridurre o prevenire le alterazioni che possono incidere sulla biomeccanica del
passo, per migliorare la resistenza e ridurre la spesa energetica, di mantenere
o migliorare la velocità, per un obiettivo funzionale, e di mantenere o
migliorare il controllo posturale per ridurre le cadute, anche con ausili.
L'intervento deve essere multidimensionale, agendo sia a livello
muscolo-scheletrico, per l'aumento della forza e della flessibilità, sia a
livello di endurance ed equilibrio.

Il cammino è un'attività continua, costituita da sequenze indipendenti ma
concatenate tra loro, quindi migliorare in modo selettivo una sola sequenza non
sempre porta ad un miglioramento dell'attività in generale.

\subsection{Le cadute}
La sindrome da instabilità è prioritaria per la sanità pubblica, con cadute
frequenti, che causano o rendono più grave la disabilità, possono portare alla
morte, e hanno un grosso costo sanitario, sociale ed economico.

La caduta è l'improvviso, non intenzionale e inaspettato abbassamento, dalla
posizione ortostatica, assisa o clinostatica. Porta a urtare col corpo al suolo.

Si classificano in accidentali, fisiologiche imprevedibili, o fisiologiche
prevedibili, molto frequenti in ospedale, in soggetti che presentano uno o più
fattori di rischio. Le conseguenze delle cadute sono traumi, fratture, e paura
di cadere di nuovo. 

Le cause sono molte, e si dividono in intrinseche ed estrinseche. Le intrinseche
sono, ad esempio, la debolezza, i disturbi di ossa e articolazioni, quelli
neurologici, sensoriali, cognitivi, l'uso di ausili, e problemi gravi ai piedi.
Quelli estrinseci sono i farmaci, combinati o a rischio (come gli
antipertensivi, che causano caduta di tensione e incontinenza, neurolettici, che
provocano disturbi extrapiramidali, ipnotici, narcotici, che calano la
vigilanza, e ipoglicemizzanti), gli ambienti domestici inadeguati e pericolosi
(con superfici scivolose, bagnate, mobili troppo alti, tappeti, dislivelli, e
oggetti in disordine), le calzature non adeguate, e gli ambienti esterni non
adeguati.

La riduzione della forza porta a cadute se associata con altri fattori, la
riduzione del ROM anche, soprattutto in associazione a modifiche delle dita,
e la riduzione della propriocezione anche, sopratutto con calzature non adatte.
L'aumento dei tempi di reazione non costituisce un problema particolare, mentre
la riduzione della prontezza muscolare è decisiva, soprattutto in situazioni in
cui possa essere importante fare un passo più lungo restando stabile.

\subsubsection{La prevenzione}
La prevenzione si mette in atto individuando i fattori di rischio, ed eseguendo
un intervento personalizzato per far recuperare la forza, l'equilibrio, la
sicurezza e la resistenza, riducendo la paura di cadere, anche con psicoterapia,
se necessario. In particolare, è importante educare il paziente e chi se ne
prende cura, per modificare l'ambiente di vita e le calzature. Spesso anche se
un anziano è integro dal punto di vista cognitivo, si vergogna a chiedere aiuto,
e per questo adotta atteggiamenti non sicuri. La valutazione del domicilio
permette di suggerire ausili, come maniglie, alza water e corrimano, e
atteggiamenti, come la rimozione dei tappeti e il posizionamento di luci
notturne, che possono aiutare.
In caso di problemi cognitivi spesso si ricorre alla contenzione, che non è una
soluzione adeguata, in quanto spesso aumenta l'agitazione, la depressione, e il
rischio di lesioni ottenute cercando di liberarsi. L'obiettivo quindi è
diminuire i fattori di rischio, e aumentare la sorveglianza, in modo da limitare
la contenzione a casi straordinari.

\subsubsection{I test}
Per valutare il cammino si usano strumenti che lo valutano qualitativamente e
quantitativamente, nella velocità (permettendo di valutare anche l'apparato
cardiocircolatorio e respiratorio, e le capacità cognitive in dual tasking col
cammino. 

Il test di Tinetti si può somministrare agli anziani, anche con lievi problemi
cognitivi, osserva le performance ed identifica i soggetti a rischio di caduta.
Dura 8-10 minuti, può essere somministrato da personale medico o sanitario, e
non ha bisogno di un setting specifico. Questo test riproduce i cambiamenti di
posizione, le manovre di equilibrio e gli aspetti del cammino che servono per
svolgere le ADL. Si divide in equilibrio, con 9 prove, che analizzano la
stazione seduta, i passaggi posturali e quella eretta, e andatura, con 7 prove,
che analizzano il cammino. In queste prove il punteggio va da 0 (non capace) a
2 (capace senza adattamenti), ma in alcune da 0 a 1. Il punteggio totale va da 0
a 28.

Per eseguire il test si usa una sedia senza braccioli. Nella sezione equilibrio,
il soggetto viene valutato da seduto, su una sedia rigida e senza braccioli, e
si valuta come esegue i movimenti, non quanto in fretta. Nella sezione andatura,
il soggetto sta in piedi davanti all'esaminatore, e cammina in un corridoio o in
una stanza, per più di 10 passi. Può usare gli ausili a cui è abituato. Se il
punteggio è inferiore a due, il soggetto non deambula, se è tra 2 e 19 deambula,
ma con alto rischio di cadute, se è tra 20 e 24 il rischio è moderato, se
maggiore di 24 è lieve, e se è 28 non ci sono rischi. Se nella sezione
equilibrio il punteggio è sotto 11, c'è un alto rischio di caduta per
l'equilibrio, mentre se nella sezione andatura è sotto 8, questa è ad alto
rischio. Secondo alcuni autori, la sezione andatura offre pochi vantaggi
rispetto alla sola sezione equilibrio, e non varrebbe la pena di somministrarla.

I problemi di questo test sono che non si riesce a dare maggiore peso ad un
item, o ad una sezione, che i pazienti con buone prestazioni vengono uniformati,
e che non si valutano adeguatamente i pazienti con Parkinson o demenza avanzata.

La scala di Berg è molto utilizzata. Analizza i cambi di posizione, le manovre di
equilibrio, e alcune ADL. \'E composto da 14 item, valutati da 0 a 4. Sopra i 45
punti si ha un buon equilibrio, sotto no, e sotto i 35 si consiglia un ausilio.

Si fa la valutazione usando un cronometro, un righello, un lettino, una sedia
senza braccioli e uno sgabello, in un tempo da 10 a 20 minuti.

I limiti di questo test sono che l'operatore deve essere bene addestrato, non è 
adatto ad analizzare i soggetti con demenza, e il tempo di somministrazione è più
lungo.

Il Timed Get Up And Go Test valuta il tempo che il paziente impiega per alzarsi
da una sedia senza braccioli, camminare in linea retta per tre metri, girarsi,
tornare alla sedia e sedersi. Emerge una forte relazione tra la velocità del
cammino e il rischio di perderne la funzione, e, in seguito, morire. Questo test
valuta solo gli aspetti temporali, mentre il testi di Tinetti valuta solo gli
aspetti qualitativi del cammino.

\paragraph{Take home messages}
L'equilibrio è una capacità complessa, che dipende dall'integrazione del sistema
muscolo-scheletrico, neurale e sensitivo.
Invecchiando, aumenta il rischio di cadere, con una genesi polifattoriale, che
include le modifiche ai vari sistemi, ma non solo.
L'intervento per rieducare l'equilibrio deve essere sistemico e polifattoriale.
Il cammino è un'attività complessa, che dipende da vari sistemi, come l'equilibrio.
Invecchiando, le funzioni fisiologiche del cammino si modificano.
I problemi dei vecchi nella deambulazione sono un segno predittivo importante.
L'andatura senile è un segno di compromissione generalizzata del SNC.
Le modificazioni fisiologiche e patologiche del cammino aumentano il rischio di
cadere.
I vari test di valutazione dell'equilibrio analizzano vari aspetti, il Tinetti è
il migliore.
Il trattamento deve essere sistemico e polifattoriale.
La contenzione diminuisce il rischio delle cadute, ma non il rischio di danni.

\section{Fratture di femore}
\subsection{Slide}

Un terzo delle donne e un quinto degli uomini oltre i 50 anni vanno incontro a
una frattura dovuta all'osteoporosi. Le fratture del bacino, nei prossimi 50
anni, dovrebbero più che raddoppiarsi. L'incidenza delle fratture del bacino è
salita moltissimo dal 2001 in Europa, con incrementi particolarmente pronunciati
in Spagna, Regno Unito e Austria. L'impatto delle fratture osteoporotiche sul
bilancio sanitario è maggiore di quello del tumore al polmone o alla prostata,
e si avvicina a quello degli ictus.

Per il trattamento si può fare una sintesi a cielo chiuso, con viti particolari,
che ha i vantaggi di preservare l'articolazione naturale, essere poco invasiva
ed economica, ma come svantaggi fa rischiare la pseudoartrosi e la necrosi
della testa, e obbliga ad un allettamento di un mese e a tre mesi di scarico.

In alternativa le endoprotesi permettono un rapido recupero, con un allettamento
breve, ma l'intervento è più lungo, con maggiore perdita di sangue, ci possono
essere numerose complicanze, e l'articolazione è artificiale.

L'ulteriore alternativa, rappresentata dal chiodo gamma, è poco invasiva e
permette il carico precoce, ma come svantaggio ha il cut-out della vite.

Il 20\% delle fratture risulta in disabilità nella deambulazione, il 60\%
in un recupero funzionale non al livello precedente alla frattura, e
dal 10\% al 20\% dei pazienti vanno incontro a morte entro un anno.

La frattura, assieme alla fragilità del paziente, e alla probabile comorbidità,
ci obbliga a considerare un tratamento multidisciplinare. Allo stesso tempo,
la fragilità, che rende più complesso il processo di cura, che non si può
realizzare se non si risolvono i marcatori della fragilità, come malnutrizione,
instabilità clinica, delirium, \dots, rende necessaria l'ortogeriatria, il
modello più appropriato per questi pazienti.

La valutazione del fisioterapista analizza lo \textbf{stato funzionale prima
dell'evento}, il tipo di evento e di intervento, lo \textbf{stato funzionale
dopo l'intervento}, il dolore, il vissuto emotivo, la collaborazione, e le
eventuali comorbilità muscolo scheletriche e sistemiche.

Nell'analisi dello stato funzionale si devono controllare le capacità di
effettuare spostamenti a letto, i passaggi posturali e i trasferimenti, la forza
dei muscoli e l'escursione articolare.

Gli obiettivi del trattamento sono il recupero della massima autonomia
funzionale e la migliore qualità di vita. L'intervento deve essere verificato,
non su un unico outcome finale, ma considerando un andamento che permetta di
tracciare lo sviluppo della situazione (usando l'indicatore Barthel) da prima
dell'evento, fino a dopo, in modo da consegnare un indicatore corretto a chi
proseguirà il progetto dopo la dimissione.

La prognosi delle fratture, a breve e a lungo termine, sulla funzione e sulla
mortalità, viene influenzata da vari fattori, come il sesso, l'età, la salute
prima dell'evento, la funzione prima dell'evento, lo stato cognitivo, il dolore,
la sede delle fratture, l'anemia, l'immobilità post operatoria, la forza
muscolare e la paura delle cadute.

La cumulated ambulation score è una scala funzionale, che valuta le performance
del soggetto, e si usa in ambito ospedaliero, precocemente.
Deve essere somministrata per tre giorni dopo l'intervento, e in base al
punteggio permette di capire se un paziente sarà da seguire anche al termine del
ricovero. 

Il trattamento del fisioterapista si basa su un razionale mirato a recuperare
caratteristiche fisiche, come ROM, forza ed elasticità, funzioni, come le ADL, e
la motricità globale, come il cammino e l'equilibrio. Purtroppo questo razionale
non è ancora stato sintetizzato in una linea guida, come dice una review
cochrane, e questo porta al fiorire di protocolli anche molto diversi tra loro: 
su 19 esperimenti analizzati, non se ne sono trovati due abbastanza simili da
essere confrontati, e questo non permette di stabilire quale sia il tipo di
esercizio migliore da applicare.

L'intervento riabilitativo, dal primo al settimo giorno dopo l'operazione,
quindi ancora in fase acuta, si deve caratterizzare con una mobilizzazione e una
verticalizzazione precoce, che permettono di ottenere una deambulazione migliore
a due mesi dall'intervento. Dal primo giorno, il paziente deve già essere messo
seduto a bordo letto, sempre nel rispetto del dolore, e associando alla
mobilizzazione degli esercizi isometrici, per il quadricipite, il gluteo, e i
gruppi di abduttori e adduttori. Questi esercizi devono essere eseguiti con
poche ripetizioni, ma eventualmente con più sedute brevi durante il giorno.

Dall'ottavo giorno alla seconda-quarta settimana, viene considerata fase
subacuta, e in questa fase si procede con un training ai vari passaggi
posturali, al cammino e alle altre ADL. Bisogna procedere con l'addestramento
dei caregiver e con il massimo rinforzo dei muscoli, per quanto possibile.

Durante l'intervento in fase sub-acuta si deve fare attenzione ai disturbi
delle funzioni esecutive, alla depressione, al delirium, e alle varie cadute
che possono avvenire durante la degenza. I disturbi delle funzioni esecutive
possono presentarsi soprattutto durante la gestione degli ausili, e hanno un
importante effetto per il risultato finale del percorso riabilitativo. Questi
problemi, quindi, vanno rilevati e segnalati precocemente. I sintomi depressivi,
poi, sono presenti in 1-4 pazienti su 10, e, oltre ad essere un ostacolo per il
percorso riabilitativo, sono anche correlati con la morte prima di un anno dopo
l'evento, se si presentano assieme alla demenza. Il delirium e le cadute in
degenza possono influenzare l'outcome: per questo vanno colte e segnalate
variazioni di attenzione, orientamento, pensiero astratto, memoria e
comportamento che potrebbero farci pensare a uno stato di confusione. Le cadute
e la paura di cadere influenzano negativamente il trattamento.

Dalla quarta settimana in poi, entriamo nella fase degli esiti: gli esercizi
riguardano prevalentemente l'equilibrio e la coordinazione, cercando di ridurre
la paura di cadere. In questa fase si deve tornare all'attività precedente alla
frattura.

Nei pazienti depressi o dementi la motivazione verso la riabilitazione è minore.
Questi problemi, per le difficoltà a relazionarsi che portano, tendono anche a
portare il terapista a diminuire l'intensità, sia per i tempi, sia per i modi,
con, alla fine, un lavoro globale migliore che viene fatto con i pazienti, e un
outcome peggiore rispetto ai pazienti che non presentino demenza o depressione.
Gli outcome peggiori influenzano negativamente la motivazione del paziente a
proseguire il trattamento, con un circolo vizioso che si viene ad instaurare.
Per questo, si suggerisce che pazienti simili vengano trattati solo da terapisti
con una formazione specifica, in modo da avere una comunicazione migliore
e un outcome migliore.

\paragraph{Take home messages}
Dopo una frattura di femore, il 60\% dei pazienti non recupera il livello
funzionale precedente alla lesione. La fragilità ossea correla con la fragilità
globale del soggetto, e l'intervento riabilitativo deve essere inserito in un
protocollo globale individualizzato. Non ci sono protocolli Evidence Based
Medicine, ma da vari studi emerge che è importante l'intensività del
trattamento. In particolare, questo correla con i pazienti dementi o depressi,
che sono a rischio di un trattamento non intensivo.

\subsection{Appunti}
Le fratture di femore sono molto frequenti, e sono causate dall'osteoporosi: se 
dopo una caduta senza forze particolari il femore si frattura, è un segno 
conclamato di osteoporosi.

Dopo le fratture, è importante tenere basso il periodo di allettamento, per 
prevenire tutti i problemi collegati. Nel recupero normale, si consiglia almeno 
un mese di allettamento e tre mesi di scarico totale. Questo non è possibile 
nell'anziano. Nelle protesi, sono positivi i brevi tempi di recupero e tempi di 
allettamento, mentre è negativo il fatto che l'intervento tende ad essere 
demolitivo (dipende anche dall'accesso), dura molto, con perdita ematica e 
conseguente anemizzazione, che può portare a \textit{delirium}, solo 
nell'anziano. L'utilizzo del chiodo gamma è positivo, sempre per il discorso 
dell'allettamento più breve possibile.

Gli outcome della frattura sono:
\begin{itemize}
\item Un quinto ha un esito di disabilità nell'ambito della deambulazione.
\item Tre quinti, oltre alla deambulazione, perdono anche la continenza, la 
capacità di vestirsi, l'orientamento spazio temporale. Questi problemi sono 
caratteristici del paziente anziano, in cui i deficit coinvolgono anche sistemi 
funzionali indipendenti.
\item Dal 10 al 20 percento va incontro a morte entro l'anno dal trauma.
\end{itemize}
Spesso, a causa dell'allettamento e della riduzione della deambulazione, dalla 
frattura del femore si va incontro a una sindrome da immobilità, e quindi il 
paziente finisce per morire per polmonite o per le altre complicazioni.

\'E importante una corretta valutazione, soprattutto della condizione prima del 
trauma, perché ovviamente non si può pensare di dare al paziente, dopo 
l'intervento, delle capacità che non aveva nemmeno prima. La valutazione deve 
tenere in gran conto anche lo stato funzionale post operazione, con il dolore, 
lo stato emotivo, e le altre condizioni. L'obiettivo del trattamento è sempre 
la migliore qualità di vita possibile, non per forza la migliore deambulazione 
possibile. Al termine dell'intervento, comunque, la valutazione deve essere non 
puntiforme, ma deve tenere conto dell'andamento globale, perché, soprattutto 
nei pazienti geriatrici, l'andamento spesso è altalenante.

L'età e il sesso maschile sono fattori prognostici negativi, mentre un buono 
stato di salute pre intervento è un fattore prognostico positivo. Il dolore 
post operatorio è anche questo un indice prognostico, per la conservazione 
della funzione. L'immobilità post operatoria e la forza muscolare hanno un 
impatto soprattutto sul recupero a breve termine, mentre non hanno grandi 
impatti a lungo termine, e nemmeno sulla mortalità.

Lo stato funzionale del paziente è un forte indicatore della sua salute: se il 
paziente si muove bene, sta bene, mentre se si muove male, è facile che abbia 
altre complicanze di salute.

Esiste un indice multidimensionale complesso, un test chiamato 
\textit{Cumulated Ambulation Score}, il cui punteggio va da 0 a 18, valutando 
alcune funzioni, come i trasferimenti da supino a seduto, i passaggi posturali, 
e la deambulazione. Il punteggio 0 indica l'incapacità del paziente, 1 che esegue 
il compito con assistenza, 2 in autonomia. Si esegue il test nei primi tre 
giorni dall'intervento, sommando i punteggi dei vari giorni. Il punteggio 
totale, che arriva a 18, se è maggiore di 10 indica una buona possibilità di 
recupero, ed è quindi l'indicatore di una buona possibilità di investimento nel 
paziente.

Con i pazienti dementi, spesso, tutti gli esercizi che vorremmo fare non si 
possono eseguire, e spesso l'unico esercizio che si riesce a fare è il 
training del cammino. Dalle revisioni Cochrane emerge come la verticalizzazione 
e il cammino siano già utili, piuttosto che niente. 
