\chapter{Concetti generali}
\paragraph{ICF}
L'ICF nasce nel 2001, e rappresenta uno standard internazionale. La sua
particolarità è di misurare sia la salute, che la disabilità, e di utilizzare un
approccio globale, adeguato per le malattie croniche e degenerative. I
principali vantaggi di avere un modello simile sono la grande comunicabilità tra
professionisti diversi, avendo uno standard a cui riferirsi, e l'attenzione, che
viene spostata dai deficit alle potenzialità del paziente. La disabilità viene
intesa come una discrepanza tra le richieste dell'ambiente e le prestazioni
offerte dall'individuo, e può dipendere da vari fattori. Lo scopo dei nostri
(e non solo nostri) trattamenti è quello di eliminare gli ostacoli che
impediscono la migliore interazione del soggetto con l'ambiente, che impediscono
la migliore qualità di vita possibile.

L'ICF è andato a sostituire la precedente classificazione di menomazioni,
disabilità e handicap, che puntava l'attenzione sulle menomazioni, che
provocavano disabilità, e, da lì, limitazioni delle attività normali, e
svantaggi per l'individuo. Siamo quindi passati da un'ottica negativa, che si
concentrava su quello che mancava, ad un'ottica positiva, che evidenzia la
salute e le potenzialità residue.
La disabilità, prima, era considerata una caratteristica della persona, che
andava trattata con un intervento individuale dei professionisti sanitari.
Questo modo di intendere la malattia era poco efficace nei confronti della
popolazione anziana, che si trova ad essere in uno stato costante di disabilità,
a causa delle problematiche legate all'età.
\\* Il nuovo modello, invece, permette di descrivere molto bene le condizioni
di salute della popolazione anziana, che possono essere molto buone anche con
delle lesioni importanti, se la rete di supporto attorno a loro è buona, e se
ci sono delle buone condizioni economiche. Al contrario, con una rete sociale
poco sviluppata, o con delle cattive condizioni economiche, anche una lesione
lieve può portare ad una bassa qualità di vita.

La valutazione ICF si basa su un modello ad albero, e prende in considerazione
vari aspetti:
\begin{itemize}
\item Aspetti correlati all'individuo
	\begin{itemize}
	\item Strutture corporee, come muscoli, stomaco, \dots Queste vengono mantenute
	dal fisioterapista.
	\item Funzioni sostenute dalle strutture, come contrazione muscolare e
	digestione, mantenute anche queste dal fisioterapista.
	\item Attività che l'individuo compie, grazie alle funzioni integre, come
	correre e mangiare. Queste vengono mantenute in collaborazione dal
	fisioterapista e dal terapista occupazionale.
	\item Partecipazione, coinvolgimento nelle varie situazioni, come per esempio
	guidare, o andare a cena fuori. Queste sono mantenute dal terapista
	occupazionale.
	\end{itemize}
\item Fattori contestuali
	\begin{itemize}
	\item Fattori ambientali, sia individuali, come la casa, il luogo di lavoro e i
	famigliari, sia sociali, come il contesto sociale e i servizi pubblici.
	\item Fattori personali, come l'etnia, l'età, che influenzano lo stato di
	salute, e anche se non sono classificati, vengono valutati.
	\end{itemize}
\end{itemize}

Come già detto, si può invecchiare in vari modi diversi:
\begin{itemize}
\item Usual aging, quando a più di 70 anni si hanno meno di due patologie
croniche in atto, e la persona è lucida, mobile, continente e con una buona
capacità visiva. La capacità psicofisica della persona si riduce
progressivamente, ma non per effetto delle malattie, che non ci sono.
\item Successfull aging, quando a più di 90 anni non ci sono alterazioni
significative delle capacità fisiche e psichiche. Queste persone sono dei casi
eccezionali.
\item Invecchiamento legato alla malattia, in cui nel paziente con più di 80
anni ci siano più di due patologie croniche in atto, e ci sia dipendenza fisica,
instabilità posturale, ipocinesia, problemi di continenza, depressione e
problemi cognitivi.
\end{itemize}

La qualità della vita di un paziente con patologie croniche viene influenzata da
fragilità, disabilità, e comorbilità (\textit{aggiungerei anche dall'ambiente e
dal supporto, ma forse sono incluse nella fragilità}).

La riabilitazione dell'anziano è diversa da quella del resto della popolazione,
perché la popolazione anziana è più varia, infatti le patologie si sommano in modi 
inaspettati, i pazienti sono più fragili (vicini alla soglia di tracollo) e
complessi, il che ci dà una maggiore instabilità clinica, e hanno poche riserve
funzionali, proprio a causa dell'età. Inoltre, i pazienti hanno una ridotta
spinta motivazionale, e hanno bisogni e aspettative diversi da quelli dei
pazienti giovani.

I pazienti geriatrici, quindi, sono \textbf{complicati}, nel senso che si
sommano diverse patologie, con uno stato clinico di \textbf{polipatologia} (le
patologie associate hanno un effetto sullo stato funzionale) e di
\textbf{comorbilità} (sono presenti due o più patologie nello stesso soggetto),
\textbf{complessi}, nel senso che le patologie presenti interagiscono tra loro,
e interagiscono con altri fattori (cognitivi, psichici, funzionali, culturali)
e \textbf{fragili}, nel senso che hanno poche riserve funzionali, portando a
incapacità di rispondere alle richieste dell'ambiente, se varia l'omeostasi.
I pazienti complicati, comunque, hanno una patologia principale da trattare,
relativamente facile da capire, e il nostro intervento si concentra su questa
patologia, e sul paziente.
Nei pazienti fragili e complessi, invece, è più difficile anche capire come
elaborare il nostro intervento, ma anche quale sia il problema principale, e per
questo, il nostro intervento non si concentra solo sul paziente, ma anche su chi
gli sta attorno, come i \textit{caregiver}, e sull'ambiente, con l'abbattimento
delle barriere. Nei pazienti complessi o fragili, inoltre, aumenta anche il
coinvolgimento del team multidisciplinare, composto sia di specialisti, sia di
professionisti sanitari.

Un problema o un paziente com-plicato si può vedere come se avesse molte pieghe,
e se andasse quindi analizzato, una piega alla volta, per risolverlo.
\\ Un problema complesso, invece, è intrecciato: per questo non basta
analizzarne una parte alla volta, ma bisogna analizzarlo nel suo insieme, con un
approccio globale.
\\ Un paziente fragile ha poca resistenza, poca capacità di rispondere alle
sollecitazioni, è come se fosse rigido.

La definizione di anziano fragile non è semplice e univoca: normalmente ci si
affida alla pratica clinica per capire e trattare le persone fragili, ma una
definizione più chiara è utile per lo studio del fenomeno, per l'identificazione
e per fornire una terapia appropriata e corretta.
Un paziente si dice fragile se presenta almeno tre tra i seguenti
\begin{itemize}
\item Segni
	\begin{itemize}
	\item Osteopenia-sarcopenia
	\item Rallentamento del cammino
	\item Debolezza della presa
	\end{itemize}
\item Sintomi
	\begin{itemize}
	\item Astenia
	\item Affaticabilità
	\item Anoressia
	\item Paura di cadere
	\end{itemize}
\end{itemize}

Per gestire il paziente fragile è necessario fare attenzione a come le patologie
influenzano la sua vita, ed essere in grado di misurare anche piccole variazioni
positive.

In particolare, le patologie possono essere nuove, o possono essere patologie
croniche che tornano a presentarsi in stato acuto, con sintomi non specifici per
la malattia che viene analizzata o per l'organo colpito: i sintomi, infatti,
si riassumono in una disabilità che provoca un crollo delle funzioni, con
difficoltà di attenzione, perdita dell'equilibrio, incontinenza, disturbi della
coscienza, peggioramento dello stato cognitivo, perdita di peso ed eventualmente
dell'omeostasi.

Le piccole variazioni che vanno apprezzate sono quelle prodotte dall'intervento
che applichiamo: non puntiamo alla risoluzione della malattia cronica, ma
cerchiamo di raggiungere un nuovo equilibrio. Le piccole modificazioni che
creiamo possono avere degli effetti più ampi sul soggetto. Per apprezzare al
meglio queste variazioni contenute, è bene esaminare le traiettorie del
paziente in generale, piuttosto che i singoli aspetti.

La valutazione del fisioterapista si basa sullo stato funzionale prima
dell'evento che ha portato al tracollo, sui dati anamnestici disponibili,
sullo stato funzionale al momento della valutazione, sul dolore, lo stato
cognitivo, la comorbilità, il vissuto emotivo del paziente, e le risorse sociali
e ambientali alle quali ha accesso. 

Il progetto riabilitativo deve raccogliere dati sul passato e sul presente del
paziente, e fornire delle previsioni su come si vorrebbe fare evolvere la sua
situazione. Per questo, si articola in:
\begin{itemize}
\item Raccolta dati
\item Definizione dei problemi riabilitativi e della loro importanza
\item Individuazione dei bisogni del paziente
\item Definizione degli obiettivi dell'intervento, con l'analisi delle risorse
e la definizione delle priorità
\item Definizione di modi e tempi per la verifica dei risultati
\item Verifica dei risultati
\item Follow-up
\end{itemize}

L'obiettivo della riabilitazione geriatrica non è una soluzione completa della 
malattia, ma piuttosto un mantenimento o un recupero della massima autonomia
possibile, con un incremento, o perlomeno un calo rallentato, della capacità di
vita della persona.

La visione con cui si effettua la valutazione dell'anziano non deve essere multi
professionale, in quanto avremmo solo una somma delle diverse competenze, ma
piuttosto multidimensionale, in modo da avere una integrazione delle varie
competenze, e una valutazione più completa.

La valutazione multidimensionale permette di valutare la salute dell'anziano nel
complesso, e d identificare il danno, e le capacità residue nei vari ambiti, con
degli strumenti condivisi tra i vari professionisti.
La valutazione multidimensionale usa i dati che vengono raccolti per poter
stabilire delle gerarchie di importanza nella cura e nell'intervento,
permettendo di definire obiettivi a breve, medio e lungo termine, e problemi
principali e secondari. Permette anche di ridurre i vari casi a uno schema, per
rendere più facile il monitoraggio nel tempo.
Questo tipo di valutazione rende più semplice il lavoro in équipe, e consente
una visione comune dei problemi e una trasmissione dei dati e dei messaggi,
rendendo più facile un intervento coordinato.

La valutazione multidimensionale ci permette di aiutare il soggetto anziano, di
curarlo al meglio, di intervenire sulla sua qualità di vita, e di garantire a
chi se ne prende cura un giusto grado di informazione e conoscenza.

Lo stato di salute dell'anziano è determinato da vari aspetti, su cui il
fisioterapista può intervenire in modi diversi:
\begin{itemize}
\item La malattia
\item Il livello cognitivo: intervento con riabilitazione cognitiva
\item Il supporto psicosociale
\item La rete sociale a disposizione: educazione dei caregiver
\item Le condizioni economiche
\item L'ambiente: adattamento dell'ambiente
\item Lo stato funzionale: riabilitazione motoria
\end{itemize}

Per curare al meglio il paziente anziano, bisogna svolgere la valutazione multi
dimensionale, lavorare in équipe nella cura, e integrare i vari servizi
disponibili, che sono:
\begin{itemize}
\item Il medico di famiglia
\item La guardia medica
\item Il centro diurno
\item Il distretto socio-sanitario, a cui gli altri sono correlati
\item Le strutture intermedie
\item L'assistenza domiciliare integrata
\item le strutture residenziali per persone non autosufficienti
\end{itemize}

La valutazione multidimensionale si svolge attraverso la scheda SVaMA (Scheda
Valutazione Multidimensionale Anziano), che passa attraverso
\begin{itemize}
\item La valutazione sanitaria, con l'anamnesi, le varie comorbidità, l'analisi
del fabbisogno di assistenza.
\item La valutazione funzionale, con le scale Barthel, Barthel mobilità, SPMSQ,
l'indice di Exton Smith e la valutazione delle ADL.
\item La valutazione: sociale, della situazione abitativa, e dell'attivazione
della rete assistenziale, controllando anche quanto efficacemente sono coperti
i bisogni della persona
\item La valutazione dell'unità valutativa multidimensionale del distretto, che
controlla il potenziale residuo della persona e ne definisce il profilo e il
progetto riabilitativo.
\end{itemize}

Questa valutazione permette di accedere alla rete dei servizi, e viene
effettuata dall'apposita unità valutativa. Lo scopo della valutazione è quello
di poter fornire all'anziano i servizi più adeguati a coprire i suoi bisogni e
a sfruttare il suo potenziale di recupero.

Il lavoro in équipe è sostanzialmente diverso da quello in gruppo: in équipe
si lavora tutti assieme, sono tutti sullo stesso piano, sia come importanza che
come responsabilità, e si condivide uno stesso obiettivo, comunicando nello
stesso modo, lavorando nello stesso modo, e coordinando i vari interventi.
\'E quindi necessario che ci siano buone capacità di dialogo, di ascolto e di
rispetto delle capacità personali e professionali.

Secondo Mary Tinetti, in questo periodo dobbiamo superare la visione
tradizionale della medicina, che mette al centro la malattia: non è più utile
guarire una singola malattia, che spesso interagisce con altre, soprattutto nel
paziente anziano, e bisogna tenere in maggior conto altri fattori collaterali,
come quelli psicologici, culturali e ambientali, che possono influenzare lo
stato di salute della persona. Inoltre, bisogna valutare di più la qualità
di vita di una persona, piuttosto che la sua durata, magari in un continuo stato
di bisogno.

\begin{landscape}

\begin{table}[]
\centering
\caption{Intervento individuale}
\label{ind}
\resizebox{1.5\textwidth}{!}{%
\begin{tabular}{lll}
\hline
\rowcolor[HTML]{C0C0C0}
                                   & Malattia acuta                                                                                                                                                         & Malattia cronica                                                                                                                                                                                                                                                                               \\ \hline
Pensieri guida                     & \begin{tabular}[c]{@{}l@{}}Eliminare la malattia, c'è bisogno\\ di un sapere specialistico,\\ la guarigione è possibile solo con le\\ metodiche corrette.\end{tabular} & \begin{tabular}[c]{@{}l@{}}Si deve ridurre il danno, alleviare la sofferenza, sostenere\\* le risorse a disposizione, migliorare la QoL,\\ Il risultato degli interventi non è mai certo, e gli interventi\\* non sono definitivi,\\ Non sempre è chiaro come si ottengono i risultati.\end{tabular} \\
\rowcolor[HTML]{C0C0C0}
Relazione tra Operatore e Paziente & \begin{tabular}[c]{@{}l@{}}L'operatore è lo specialista,\\ il paziente deve seguire le indicazioni.\end{tabular}                                                       & \begin{tabular}[c]{@{}l@{}}Gli operatori sono tanti, con relazioni multiple e variabili\\* tra loro, e una comunicazione fondamentale.\\ Bisogna saper ascoltare e cooperare, nel rispetto delle autonomie,\\ Le decisioni vanno prese col paziente e con la famiglia.\end{tabular}               \\
Comportamenti operativi            & \begin{tabular}[c]{@{}l@{}}Sequenza di azioni predefinite:\\ diagnosi, terapia, guarigione.\end{tabular}                                                               & \begin{tabular}[c]{@{}l@{}}Non c'è un percorso valido per tutti, gli interventi vanno\\* sempre verificati e condivisi con il paziente e i famigliari.                                                                                                                                                                      \end{tabular} \\ \hline
\end{tabular}%
}
\end{table}

\begin{table}[]
\centering
\caption{Funzionamento organizzativo}
\label{org}
\resizebox{1.5\textwidth}{!}{%
\begin{tabular}{@{}lll@{}}
\toprule
\rowcolor[HTML]{C0C0C0} 
                     & Malattia acuta                                                                                                                    & Malattia cronica                                                                                                                                               \\ \midrule
Prodotti attesi      & \begin{tabular}[c]{@{}l@{}}Tecnologie avanzate,\\ Efficacia ed efficienza\\ Attenzione al rapporto \\ costi/benefici\end{tabular} & \cellcolor[HTML]{FFFFFF}\begin{tabular}[c]{@{}l@{}}Ricerca del benessere possibile\\ Ricerca di una relazione positiva\\ Monitoraggio delle spese\end{tabular} \\
\rowcolor[HTML]{C0C0C0} 
Destinatari fruitori & Singoli soggetti malati                                                                                                           & \begin{tabular}[c]{@{}l@{}}Pazienti, famiglie, operatori,\\ servizi pubblici e privati, amministrazioni locali,\\ organizzazioni di volontariato,\end{tabular} \\
Divisione lavoro     & \begin{tabular}[c]{@{}l@{}}Divisione per specializzazione\\ e gerarchia\end{tabular}                                              & \begin{tabular}[c]{@{}l@{}}Lavoro in équipe, per processi e con\\ coordinamenti funzionali tra le parti.\end{tabular}                                          \\ \bottomrule
\end{tabular}%
}
\end{table}

\end{landscape}

\paragraph{Take home messages}
\begin{itemize}
\item L'ICF è una classificazione che segue il modello biopsicosociale.
\item \'E diverso riabilitare un adulto vecchio, da riabilitare un anziano fragile.
\item L'anziano che presenti fragilità è un anziano complicato, e complesso, e
per questo, richiede un approccio multidimensionale, che esamini i vari
determinanti della salute, ad esempio usando SVaMA come strumento.
\item Il progetto di cura, che sia riabilitativo, assistenziale o sociale, deve
valutare il soggetto in tutti i tempi: sia nel passato, che nel presente, che
nel futuro.
\item L'obiettivo della cura e della riabilitazione, nell'anziano fragile, è il
mantenimento della migliore autonomia possibile, e della migliore QoL possibile.
\item La valutazione e il trattamento dell'anziano fragile non possono essere
svolti da un solo professionista, ma vanno fatti in équipe.
\end{itemize}
