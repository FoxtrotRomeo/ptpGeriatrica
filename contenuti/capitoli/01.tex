\mainmatter
\chapter{Geriatria e gerontologia}
La gerontologia è una parte della medicina, che vuole identificare i meccanismi 
biologici dell'invecchiamento, 
includendo anche gli aspetti sociali e psicologici che possano influenzare lo 
stato di salute e le malattie tipiche.
La geriatria, invece, studia le malattie dell'anziano e i loro esiti in 
disabilità, per ritardare il degrado funzionale 
e mentale, e mantenere l'autosufficienza e la qualità di vita.

La medicina generale segue lo stesso schema teorico della medicina geriatrica: 
entrambe eseguono una valutazione delle 
perdite e delle capacità residue, entrambe intervengono, o per curare le
malattie o con un intervento globale sul paziente e sull'ambiente, e mentre una
ha come scopo la guarigione, l'altra mira più al mantenimento
dell'autosufficienza. \\* L'invecchiamento viene 
considerato un processo multifattoriale, in quanto i diversi meccanismi che 
intervengono nel processo, da soli, non bastano per spiegarne la globalità. Un 
esempio può essere quello dell'\textit{inflamn-aging}, per cui il sistema 
immunitario diventa sempre più inefficiente col passare del tempo, e viene 
sottoposto a sempre più stimoli antigenici, dovuti al prolungarsi della vita. 
Per questo, risponde sempre più con risposte infiammatorie, che fanno parte dei 
meccanismi di difesa, e questo potrebbe essere correlato all'aumento di 
patologie a patogenesi infiammatoria. Questo porta, nel tempo, ad una 
condizione di infiammazione cronica.

Il processo è favorito da una predisposizione genetica, ma anche da stili di 
vita errati, come fumo, alimentazione (in particolare l'eccesso calorico), 
l'attività fisica e il consumo di alcool.

Anni fa, il grafico della distribuzione della popolazione era a forma di 
piramide, con un vertice stretto e una base allargata. Al momento, la base 
della piramide tende ad essere più stretta rispetto alla parte centrale, e nei 
prossimi anni la forma si modificherà ancora, secondo le previsioni, portando 
ad una forma simile ad un rettangolo. Questo è dato sia dall'aumentata 
sopravvivenza, sia dal diminuito tasso di natalità. Provoca un cambiamento 
nella famiglia tradizionale, e nel funzionamento della società.

L'invecchiamento può essere un processo fisiologico, di successo, o può portare 
a disabilità e dipendenza. Alla fine della vita è molto probabile che ci sia un 
periodo di disabilità, ma il nostro scopo è ridurre questo periodo al minimo, 
permettendo alle persone di invecchiare bene. Per ridurre questo periodo, è 
importante, da un lato, la predisposizione genetica, ma dall'altro è
fondamentale uno stile di vita corretto, con attività fisica, buona 
alimentazione, e senza abuso di fumo e alcool. In particolare, nel confronto 
tra generi, si evidenzia come, a partire dai 75 anni, le donne abbiano una 
aspettativa di vita più lunga degli uomini, ma allo stesso tempo sia più lungo 
il periodo in cui avranno bisogno di aiuto, e quello che vivranno con 
disabilità.

Il processo di invecchiamento legato all'apparato locomotore si caratterizza 
per:
\begin{itemize}
 \item Sarcopenia, con modificazioni del muscolo che possono essere
 \begin{itemize}
  \item Strutturali, come la riduzione del numero e della dimensione delle 
fibre, e l'aumento della massa grassa intramuscolare
  \item Funzionali, con la riduzione di forza, potenza e resistenza
 \end{itemize}
 \item Osteopenia, che è più marcata nelle donne, a causa delle modificazioni 
ormonali post-menopausa, ma è presente anche negli uomini, e viene in 
particolare influenzata dalla sarcopenia
 \item Modificazioni strutturali alle cartilagini, che portano a modificazioni 
funzionali, con diminuita resistenza alle sollecitazioni tensive. A queste 
modificazioni si associano fenomeni degenerativi artrosici.
\end{itemize}

Tutto questo, comunque, è positivamente influenzabile dall'attività fisica, 
come viene dimostrato dagli effetti che ha sugli atleti anziani.

Il processo di invecchiamento legato al SNC porta a modificazioni dell'encefalo
\begin{itemize}
\item Strutturali, come
\begin{itemize}
 \item Riduzione del numero dei neuroni, in zona frontale, temporale, 
nell'ippocampo e nei nuclei della base, che può essere limitata dal movimento
 \item Depositi non strutturati di sostanze, come la beta-amiloide
 \item Degenerazione dei piccoli vasi
\end{itemize}
\item Funzionali, come 
\begin{itemize}
\item Deficit di memoria e altre funzioni cognitive, come l'attenzione
\item Alterazioni del ritmo sonno-veglia
\item Disturbi dell'equilibrio e della fluidità del movimento
\end{itemize}
\end{itemize}

Anche questo processo di degradazione è riducibile: di recente si è dimostrato 
che il sistema nervoso si può modificare, durante la vita, sia aggiungendo 
nuovi collegamenti tra le cellule (\textit{sprouting}), sia aggiungendo nuove 
cellule nervose, o rendendo attive quelle silenti in un certo momento 
(\textit{neurogenesi}). Questi meccanismi sono influenzabili da pensiero, 
apprendimento, ed esperienze di vita, che portano a modificare il cervello e 
l'espressione genica. In particolare, l'attività fisica aumenta la perfusione 
dell'ippocampo, portando, nei roditori, ad un aumento della neurogenesi nella 
zona. In un esperimento su soggetti umani, si è notato che gli esercizi 
aerobici portavano ad una crescita del giro dentato dell'ippocampo, con un 
aumento delle prestazioni mnestiche, mentre nel gruppo di controllo si è notato 
un decrescere continuo, legato alla curva della senescenza. Questo decrescere 
era meno pronunciato in chi presentava una buona forma fisica all'inizio 
dell'esperimento.

L'invecchiamento ha effetto anche sugli organi di senso, con una riduzione 
delle capacità sensoriali, con alterazioni strutturali, anche senza patologie 
correlate, e sull'apparato endocrino, con alterazioni strutturali del 
parenchima legate alla riduzione di ormoni tiroidei, e incremento della 
resistenza all'insulina, con ridotta tolleranza al glucosio (anche questo 
aspetto migliora facilmente con l'attività fisica). Ha effetto anche 
sull'apparato nefrourinario, con alterazioni strutturali del rene, e 
alterazioni funzionali, che rendono impossibile adattarsi alle perdite 
protratte di sodio, ed eliminare il sodio in eccesso. Questo porta all'aumento 
del rischio di insufficienza renale acuta.

L'invecchiamento ha effetto anche sull'apparato cardiovascolare, con modifiche
\begin{itemize}
 \item Strutturali del muscolo cardiaco, con
 \begin{itemize}
  \item Riduzione del numero dei miocardiociti, e ipertrofia dei rimanenti
  \item Riduzione del numero di cellule pace-maker
  \item Deposito di collagene e adipe
 \end{itemize}
 \item Modifiche strutturali delle grandi arterie, con perdita dell'elasticità 
e della distensibilità delle pareti
 \item Modifiche funzionali, con riduzione della compliance arteriosa, e 
aumentata vulnerabilità al danno arterioso.
\end{itemize}

Tutte queste modifiche sono comuni anche ai principali fattori di rischio 
cardiovascolare. Nell'anziano cala la risposta cardiaca allo sforzo, rispetto 
al giovane, con minore incremento della frequenza, compensato da un maggiore 
aumento della gittata, dovuto ad un aumento del volume telediastolico. Questo 
meccanismo può essere influenzato dall'allenamento fisico: può aumentare la 
frazione di eiezione del ventricolo, con un allenamento fisico prolungato. Cala 
anche la responsività beta-adrenergica, causando una ridotta performance in 
caso di sforzo massimale.

Sull'apparato respiratorio, l'invecchiamento agisce con modifiche strutturali 
della gabbia toracica, sia sui diametri, sia aumentando la cifosi dorsale e 
facendo calare gli spazi intercostali. Questo porta ad una modifica funzionale 
dei muscoli respiratori, facendone calare la forza e la resistenza.
Ci sono anche delle modifiche strutturali e funzionali delle vie respiratorie, 
mostrando una ridotta capacità di clearance mucociliare, e una progressiva 
dilatazione delle vie aeree. Inoltre, cala la riserva respiratoria e 
l'elasticità del parenchima. Ci sono anche delle modifiche funzionali dei 
volumi e dei flussi polmonari, ma lo scambio gassoso, nell'anziano, resta 
adeguato, in assenza di patologie. Ovviamente, tutti questi parametri vengono 
influenzati molto dal fumo.

L'anziano ha una modalità caratteristica per ammalarsi: le sindromi. Le 5 più 
diffuse sono:
\begin{itemize}
 \item Sindrome da immobilità
 \item Sindrome da instabilità
 \item Sindrome da incontinenza urinaria
 \item Sindrome da impairment cognitivo
 \item Sindrome da reazione ai farmaci
\end{itemize}

Rispetto alle malattie, le sindromi non hanno segni e sintomi specifici: 
portano ad effetti che non sembrano direttamente correlati o correlabili con la 
sindrome in atto, e non hanno un determinante unico, ma ne hanno molti, che 
interagiscono tra loro. I segni e i sintomi non sono correlati ad una specifica 
sede di lesione, e le disabilità non sono specifiche, provocando un crollo 
funzionale. L'approccio, quindi, non deve essere analitico, ma 
multidimensionale. Le sindromi sono, in genere, un insieme di segni e sintomi 
che rappresentano le manifestazioni di una o più malattie, indipendentemente 
dall'eziologia di queste malattie.

\subsection{Sindrome da immobilità}
\'E una condizione che interessa più organi e apparati, provocata 
dall'immobilità, che porta a sintomi e segni diversi da quelli delle cause che 
hanno portato all'immobilità.

Questa sindrome parte da un evento immobilizzante, che fa superare, 
all'anziano, la \textit{soglia di riserva funzionale}, e il soggetto non riesce 
più a tornare allo stato precedente all'evento. In generale, nel paziente 
fragile, il superamento della soglia porta a peggioramenti improvvisi dello 
stato di salute, spesso irreversibili, fino anche alla morte.
I danni ad organi ed apparati sono causati dall'immobilità, dall'aumentato 
tempo passato in posizione orizzontale, e dalla perdita di afferenze 
sensoriali.
C'è una grande variabilità, da soggetto a soggetto, nella velocità con cui 
occorre la sindrome e nella sua gravità. Questi fattori sono influenzati anche 
dall'evento che provoca l'immobilizzazione.

Gli apparati coinvolti sono:
\begin{itemize}
 \item Cardiocircolatorio: porta ad una minore efficienza cardiaca, minore 
pressione arteriosa, maggiore rischio di trombosi venose e di embolia 
polmonare, e minore capacità fisica.
 \item Respiratorio: porta a riduzione della funzionalità respiratoria, stasi 
delle secrezioni e aumentato rischio di infezioni polmonari.
 \item Muscolare: porta a modificazioni della struttura e della funzione dei 
muscoli, particolarmente nei muscoli antigravitari, ad una perdita di trofismo, 
e ad un accorciamento muscolare.
 \item Articolare: porta ad una riduzione del ROM, ad un aumento di rigidità e 
retrazioni tendinee, legamentose e capsulari, e può portare anche all'anchilosi.
 \item Osseo: porta a una perdita di massa ossea (con osteopenia e 
osteoporosi), con aumentato rischio di fratture patologiche.
 \item Urinario: porta ad un aumentato ristagno vescicale, che a sua volta 
porta a aumento di infezioni urinarie, e diminuzione della capacità di 
percepire lo stimolo, che porta a incontinenza.
 \item Digerente: porta a un calo del senso di fame, un aumentato rischio di 
malnutrizione, e a una minore mobilità intestinale, che porta a stipsi, fino ad 
una quasi occlusione dell'intestino.
 \item Cutaneo: porta ad un aumento del rischio di lesioni da pressione.
 \item Neurologico: porta ad una minore stimolazione sensitivo-sensoriale, che 
a sua volta porta a disturbi di equilibrio e coordinazione, con difficoltà ad 
eseguire i passaggi posturali. C'è un aumentato rischio di lesioni da 
compressione dei nervi periferici, e confusione e disorientamento nello spazio 
tempo.
 \item Psichico: porta a deflessione del tono dell'umore, apatia e ritiro dal 
mondo.
\end{itemize}

Le fasi in cui si articola la sindrome da immobilità sono:
\begin{enumerate}
\item Alterazione dei meccanismi di controllo posturale: fase reversibile, con
la motilità volontaria dei vari segmenti mantenuta intatta, una discreta
motilità globale, il trofismo dei muscoli che viene mantenuto, ma un deficit nel
controllo posturale antigravitario.
\item Perdita del coordinamento e dell'iniziativa motoria: fase reversibile, con
il paziente che resta sempre in decubito supino, si muove solo se viene
stimolato a farlo, e conserva la motilità segmentaria, ma, appunto, non
volontaria, ma solo su stimolo. Viene invece compromessa la motilità globale,
c'è sofferenza di muscoli, articolazioni, apparato tegumentario e nervoso.
\item Completa immobilità, con atteggiamento in posizione fetale: fase
irreversibile, sono possibili dei lenti movimenti segmentari, con ampiezza
ridotta, solo se la stimolazione è intensa. Si ha un ipertono flessorio
generalizzato, sono ormai strutturate le rigidità articolari, sono possibili
lesioni cutanee, ed è presente un'incontinenza sia urinaria che fecale.
\end{enumerate}

L'intervento del fisioterapista, in questi casi, si articola in due fasi:
\begin{itemize}
\item Fase di allettamento: si cerca di mantenere il trofismo dei muscoli e
l'autonomia nei passaggi posturali, nella continenza e nelle varie ADL, e di
contrastare la perdita progressiva del controllo antigravitario. Si cerca anche
di prevenire le complicanze cardiache, respiratorie, cutanee e psichiche.
\item Fase di recupero: si cerca di individuare le modalità di trasferimento dal
letto alla carrozzina più idonee, e di trovare l'ausilio più adatto al paziente.
Si cerca anche di rieducare ai trasferimenti e ai cambi di posizione, e di
recuperare l'equilibrio in posizione seduta, eretta e mentre si cammina, ma
anche di recuperare il cammino e la resistenza cardiaca e respiratoria. Bisogna
anche stimolare il paziente dal punto di vista cognitivo, per permettergli di
recuperare l'orientamento nello spazio-tempo.
\end{itemize}

In queste fasi è fondamentale integrare l'intervento del fisioterapista con
quello del personale di assistenza, e con i famigliari più stretti, che si
prendono cura del paziente. Se è possibile, è bene inserire nel progetto anche
un terapista occupazionale.

\paragraph{Incontinenza urinaria} L'incontinenza urinaria ha una genesi
che dipende da molti fattori, tra cui alterazioni del tratto urinario, farmaci
errati, depressione, delirium, \dots, il che la rende una sindrome.
Spesso viene trascurata, nella valutazione e nel trattamento, ma peggiora la
qualità di vita e aumenta lo stress per chi presta assistenza.
Un'insufficienza urinaria acuta, spesso, è reversibile se se ne rimuovono le
cause, che possono essere \textit{il \textbf{d}elirium}, \textbf{i}nfezioni
varie, vaginite \textbf{a}trofica, (\textbf{P})farmaci, eccessi
\textbf{p}sicologici, \textbf{e}ccesso di fluidi, \textbf{r}estrizione,
\textbf{s}tool(?) e costipazione. Tutte queste cause assieme creano l'acronimo
\textbf{DIAPPERS}. Si può riabilitare il pavimento pelvico, se il paziente è
collaborante, o se non lo è si può procedere a modificare la situazione attorno
a lui: cateterismi, modificazioni dei vestiti, adattamento dell'ambiente sono
alcune delle soluzioni.

L'incontinenza urinaria può essere classificata in base al fatto se sia acuta e
reversibile, o cronica e persistente, in base al modo con cui si presenta, e in
base all'autonomia del paziente che la presenta. Quest'ultima classificazione
è molto utile per i pazienti in casa di riposo.

\paragraph{La reazione avversa ai farmaci} La reazione avversa ai farmaci è una
risposta ad un farmaco che procuri un danno non intenzionale, che si manifesti a
dosi che normalmente vengono utilizzate per profilassi, diagnosi o terapia.
\'E un problema diffuso nei Paesi Occidentali, con un impatto sulla salute dei
cittadini e sui costi della sanità. Gli anziani e i dementi sono le categorie
maggiormente a rischio.
La reazione avversa ai farmaci viene causata da un aumento del consumo di
farmaci, e dalle alterazioni della farmacocinetica nei soggetti che vanno
incontro alla reazione. La tossicità viene imputata al singolo farmaco, più che
all'interazione tra farmaci diversi. Per questo, la migliore regola è dare la
minima dose efficace al paziente.

\paragraph{Take home messages}
\begin{itemize}
\item La geriatria e la gerontologia studiano l'invecchiamento dell'uomo, con un
approccio olistico.
\item L'invecchiamento è un processo naturale, durante il quale cala la capacità
di adattamento dell'organismo all'ambiente.
\item L'invecchiamento può venire influenzato dagli stili di vita, come la
dieta, il fumo, l'alcool, ma soprattutto l'attività fisica.
\item Aumentando l'età aumentano i rischi di disabilità e dipendenza.
\item L'anziano si ammala soprattutto presentando sindromi, con eziologia
multifattoriale, e per questo va trattato con un approccio multifattoriale e
multidisciplinare.
\end{itemize}
